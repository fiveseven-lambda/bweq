\documentclass{ltjsarticle}
\usepackage{amsmath, amssymb, amsthm}
\usepackage{enumitem}
\usepackage{algorithm, algpseudocode}
\newcommand\NN{\mathbb{N}}
\newcommand\bw{\mathrel{\text{b.w.}}}
\DeclareMathOperator\head{head}
\DeclareMathOperator\tail{tail}
\begin{document}
空でない集合$A$に対し,
$A$の有限列全体の集合$\bigcup_{n = 0}^\infty A^n$を$A^*$と書く.
ただし$A^0$は空列$\varepsilon$のみを要素としてもつ集合$\{ \varepsilon \}$とする.
$\boldsymbol{a} = (a_1, \dotsc, a_n) \in A^*$ ($n > 0$)の長さ$n$を$\lvert\boldsymbol{a}\rvert$と書き,
$\lvert\varepsilon\rvert = 0$とする.
$\boldsymbol{a} = (a_1, \dotsc, a_n) \in A^*$ ($n \ge 2$)に対し,$\head\boldsymbol{a} = a_1$,$\tail\boldsymbol{a} = (a_2, \dotsc, a_n)$とする.
$a \in A^1 = A$に対しては$\head a = a$,$\tail a = \varepsilon$とする.
$\boldsymbol{a} = (a_1, \dotsc, a_m) \in A^*$ ($m > 0$)と
$\boldsymbol{b} = (b_1, \dotsc, b_n) \in A^*$ ($n > 0$)の連結を
$(\boldsymbol{a}, \boldsymbol{b}) = (a_1, \dotsc, a_m, b_1, \dotsc, b_n) \in A^*$と定義する.
$\boldsymbol{a} \in A^*$に対し
$(\boldsymbol{a}, \varepsilon) = (\varepsilon, \boldsymbol{a}) = \boldsymbol{a}$とする.

$\boldsymbol{a}, \boldsymbol{b} \in A^*$に対し,
2項関係$\boldsymbol{a} \bw \boldsymbol{b}$(b.w. は begins with の略)を次のように定義する:
ある$\boldsymbol{c} \in A^*$が存在して$\boldsymbol{a} = (\boldsymbol{b}, \boldsymbol{c})$であるとき,
またそのときに限り,$\boldsymbol{a} \bw \boldsymbol{b}$である.以下が成り立つ.
\begin{itemize}
  \item 任意の$\boldsymbol{a} \in A^*$について,$\boldsymbol{a} \bw \boldsymbol{a}$である.
  \item 任意の$\boldsymbol{a}, \boldsymbol{b} \in A^*$について,
    $\boldsymbol{a} \bw \boldsymbol{b}$かつ$\boldsymbol{b} \bw \boldsymbol{a}$ならば
    $\boldsymbol{a} = \boldsymbol{b}$である.
  \item 任意の$\boldsymbol{a}, \boldsymbol{b}, \boldsymbol{c} \in A^*$について,
    $\boldsymbol{a} \bw \boldsymbol{b}$かつ$\boldsymbol{b} \bw \boldsymbol{c}$ならば
    $\boldsymbol{a} \bw \boldsymbol{c}$である.
  \item 任意の$\boldsymbol{a} \in A^*$について,$\boldsymbol{a} \bw \varepsilon$である.
  \item 任意の$\boldsymbol{a}, \boldsymbol{b} \in A^*$について,
    $\boldsymbol{a} \bw \boldsymbol{b}$ならば$\lvert\boldsymbol{a}\rvert \ge \lvert\boldsymbol{b}\rvert$である.
  \item 任意の$\boldsymbol{a}, \boldsymbol{b} \in A^*$ ($\boldsymbol{a}, \boldsymbol{b} \ne \varepsilon$)について,
    $\boldsymbol{a} \bw \boldsymbol{b}$は$\head\boldsymbol{a} = \head\boldsymbol{b}$かつ$\tail\boldsymbol{a} \bw \tail\boldsymbol{b}$と同値である.
\end{itemize}

$\NN$を非負整数の集合とする.

これから,「b.w.を含む式を解く」ということについて考えたい.
以下はその例である.
\begin{itemize}
  \item $\boldsymbol{x} \in \NN^*$に対し,
    $(\boldsymbol{x}, 1, 2, 1) \bw (1, 2, 1)$は,
    以下のいずれかが成り立つことと同値である.
    \begin{itemize}
      \item $\boldsymbol{x} = \varepsilon$.
      \item $\boldsymbol{x} = (1, 2)$.
      \item ある$\boldsymbol{t} \in \NN^*$が存在して$\boldsymbol{x} = (1, 2, 1, \boldsymbol{t})$.
    \end{itemize}
  \item $\boldsymbol{x}, \boldsymbol{y} \in \NN^*$に対し,
    $(2, 1, 1) \bw (\boldsymbol{x}, 1)$かつ$\boldsymbol{x} \bw (\boldsymbol{y}, 1)$が成り立つことは,
    $\boldsymbol{x} = (2, 1)$かつ$\boldsymbol{y} = 2$が成り立つことと同値である.
\end{itemize}

このように,変数を含み「$(式) \bw (式)$」という形で表される式がいくつか与えられたとき,
それらを満たす解の集合を得る方法を,様々な場合について考える.
\section{1変数の場合}
有限集合$P, Q \subset \NN^* \times \NN^*$が与えられる.
任意の$(\boldsymbol{a}, \boldsymbol{b}) \in P$について$(\boldsymbol{x}, \boldsymbol{a}) \bw \boldsymbol{b}$を満たし,
任意の$(\boldsymbol{c}, \boldsymbol{d}) \in Q$について$\boldsymbol{c} \bw (\boldsymbol{x}, \boldsymbol{d})$を満たすような
$\boldsymbol{x} \in \NN^*$全体の集合$S(P, Q)$を求めたい.

$\boldsymbol{x} = \varepsilon$が解となることは,
任意の$(\boldsymbol{a}, \boldsymbol{b}) \in P$について$\boldsymbol{a} \bw \boldsymbol{b}$が成り立ち,
かつ任意の$(\boldsymbol{c}, \boldsymbol{d}) \in Q$について$\boldsymbol{c} \bw \boldsymbol{d}$が成り立つことと同値である.
以下,$\boldsymbol{x} = \varepsilon$以外の解$S'(P, Q) = S(P, Q) \setminus \{\varepsilon\}$を求める.

$(\varepsilon, \boldsymbol{d}) \in Q$が存在する場合,
$\varepsilon \bw (\boldsymbol{x}, \boldsymbol{d})$より
$\boldsymbol{x} = \boldsymbol{d} = \varepsilon$でなければいけないため
$S'(P, Q) = \varnothing$.
一方$(\boldsymbol{a}, \boldsymbol{b}) \in P$において
$\boldsymbol{b} = \varepsilon$の場合は,
任意の$\boldsymbol{x} \in \NN^*$が$(\boldsymbol{x}, \boldsymbol{a}) \bw \boldsymbol{b} = \varepsilon$を満たすから,
$S'(P \setminus (\NN^* \times \{ \varepsilon \}), Q) = S'(P, Q)$である.
以下,$(\boldsymbol{a}, \boldsymbol{b}) \in P$ならば$\boldsymbol{b} \ne \varepsilon$であり,
$(\boldsymbol{c}, \boldsymbol{d}) \in Q$ならば$\boldsymbol{c} \ne \varepsilon$であると仮定する.

任意の$(\boldsymbol{a}, \boldsymbol{b}) \in P$について,
$\boldsymbol{x} \ne \varepsilon$,$\boldsymbol{b} \ne \varepsilon$より,
$(\boldsymbol{x}, \boldsymbol{a}) \bw \boldsymbol{b}$は
$\head \boldsymbol{x} = \head \boldsymbol{b}$かつ
$(\tail \boldsymbol{x}, \boldsymbol{a}) \bw \tail \boldsymbol{b}$
と同値である.
同様に,任意の$(\boldsymbol{c}, \boldsymbol{d}) \in Q$について,
$\boldsymbol{c} \ne \varepsilon$,$\boldsymbol{x} \ne \varepsilon$より,
$\boldsymbol{c} \bw (\boldsymbol{x}, \boldsymbol{d})$は
$\head \boldsymbol{c} = \head \boldsymbol{x}$かつ
$\tail\boldsymbol{c} \bw (\tail\boldsymbol{x}, \boldsymbol{d})$と同値である.
従って,
\begin{itemize}
  \item $P_\text{head} = \{\head \boldsymbol{b} \mid (\boldsymbol{a}, \boldsymbol{b}) \in P\}$
  \item $P_\text{tail} = \{(\boldsymbol{a}, \tail \boldsymbol{b}) \mid (\boldsymbol{a}, \boldsymbol{b}) \in P\}$
  \item $Q_\text{head} = \{\head \boldsymbol{c} \mid (\boldsymbol{c}, \boldsymbol{d}) \in Q\}$
  \item $Q_\text{tail} = \{(\tail \boldsymbol{c}, \boldsymbol{d}) \mid (\boldsymbol{c}, \boldsymbol{d}) \in Q\}$
\end{itemize}
とおくと,任意の$h \in P_\text{head} \cup Q_\text{head}$について$\head \boldsymbol{x} \in h$が成り立ち,
かつ$\tail \boldsymbol{x} \in S(P', Q')$が成り立つとき,かつそのときに限り,
$\boldsymbol{x} \in S'(P, Q)$が成り立つ.

以上から,$P, Q \subset \NN^* \times \NN^*$を受け取って$S(P, Q)$を返す
関数\textsc{SolveSingleVariable}のアルゴリズムは以下のようになる.
\begin{algorithmic}
  \Function{SolveSingleVariable}{$P$, $Q$}
    \State $S \gets \varnothing$
    \If{
      $\boldsymbol{a} \bw \boldsymbol{b}$ for all $(\boldsymbol{a}, \boldsymbol{b}) \in P$
      \textbf{and} $\boldsymbol{c} \bw \boldsymbol{d}$ for all $(\boldsymbol{c}, \boldsymbol{d}) \in Q$
    }
      \State $S \gets \{ \varepsilon \}$
    \EndIf
    \State $P_\text{head}, P_\text{tail}, Q_\text{head}, Q_\text{tail} \gets \varnothing$
    \ForAll{$(\boldsymbol{a}, \boldsymbol{b}) \in P$}
      \If{$\boldsymbol{b} \ne \varepsilon$}
        \State $P_\text{head} \gets P_\text{head} \cup \{\head \boldsymbol{b}\}$
        \State $P_\text{tail} \gets P_\text{tail} \cup \{(\boldsymbol{a}, \tail \boldsymbol{b})\}$
      \EndIf
    \EndFor
    \ForAll{$(\boldsymbol{c}, \boldsymbol{d}) \in Q$}
      \If{$\boldsymbol{c} = \varepsilon$}
        \State \Return $S$
      \Else
        \State $Q_\text{head} \gets Q_\text{head} \cup \{\head \boldsymbol{c}\}$
        \State $Q_\text{tail} \gets Q_\text{tail} \cup \{(\tail \boldsymbol{c}, \boldsymbol{d})\}$
      \EndIf
    \EndFor
    \If{$P_\text{head} = Q_\text{head} = \varnothing$}
      \State \Return $\NN^*$
    \EndIf
    \If{$P_\text{head} \cup Q_\text{head} = \{x_1\}$ for some $x_1 \in \NN^*$}
      \State $S \gets S \cup \{ (x_1, \boldsymbol{x}') \mid \boldsymbol{x}' \in \Call{SolveSingleVariable}{P_\text{tail}, Q_\text{tail}} \}$
    \EndIf
    \State \Return $S$
  \EndFunction
\end{algorithmic}
\end{document}